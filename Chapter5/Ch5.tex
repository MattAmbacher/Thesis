\chapter{Discussion and Conclusions}
\label{ch:ch5}

\section{Discussion}
In this section, we first analyze the overall decomposition, i.e\ the results of the first 30 vertical modes summed together compared to the full-depth averaged Helmholtz decomposition. Next, we discuss the energy spectra in each of the vertical modes. Finally, some concluding remarks are given.

\subsection{Overall Analysis}
One of the first noticeable features from the geostrophic and ageostrophic decompositions in Figure \ref{fig:totalGeoAgeo} is the relatively steep slope associated with the ageostrophic mesoscale spectrum. Compared to the $-3$ and $-5/3$ slopes of the rotational and divergent kinetic energy in the literature, the summation of the first 30 vertical modes resulted in mesoscale spectra slopes of $-3.1$ and $-2.7$ for the geostrophic and ageostrophic modes, respectively. Though the geostrophic and RKE mesoscale spectra slopes are somewhat in agreement, at -3.1 and -2.7, respectively, the mesoscale slope of the ageostrophic spectrum at $-2.7$ is much steeper than the DKE spectrum at $-1.9$.\\

Before we discuss the discrepancies between the geostrophic/ageostrophic decomposition and the RKE/DKE decomposition, we turn our attention to the shallower RKE spectrum and the steeper DKE spectrum when compared to the $-3$ and $-5/3$ mesoscale slopes found in the literature. Since the baroclinic instability simulation was computed on a grid with evenly-spaced pressure levels, this translates to a clustering of points near the surface that get increasingly spaced apart higher in the atmosphere in $z$ coordinates. The strong surface fronts therefore have more impact on the energy spectrum of the instabilities, which cause shallowing in the large scales of the vortical motion. Previous research by Peng et al.\ has indeed shown shallower spectra near the ground \cite{Peng2013}. Additionally, the large Rossby number can also be causing a reduction in the large scales of vortical motion, instead transferring some of that energy to the ageostrophic component at the mesoscale (e.g.\ \cite{Bartello2010}).\\

The ageostrophic spectrum contains more total energy over the entire synoptic scale and mesoscale when compared to the DKE. This is due to the available potential energy (APE) in the ageostrophic mode. Since the baroclinic instabilities grow by converting APE to kinetic energy, at large scales we expect a large amount of potential energy as well as kinetic energy to drive the downward cascade of energy. At large scales and throughout the mesoscale, we see that the potential energy is decreasing with increasing wavenumber. At dimensionless wavenumbers of  $\tilde{k} > 200$, which corresponds to wavelengths smaller than approximately $25$ km, the ageostrophic and DKE spectra are well aligned. At these small scales, much of the APE has been converted to kinetic energy just before the numerical dissipation removes energy from the system. The RKE spectrum is similar to the geostrophic spectrum, suggesting that most of the vortical energy is in fact kinetic. At the smallest horizontal scales, the total kinetic energy in the domain (i.e.\ the RKE and DKE) appears to contain more energy than the total energy (i.e.\ geostrophic and ageostrophic modes combined), which would be unphysical. This can be explained from the exclusion of vertical modes higher than 30. Since increasing vertical mode number corresponds to decreasing equivalent depth, the energy contribution at higher vertical modes is increasingly concentrated in larger horizontal wavenumbers. We expect that by including more vertical modes, it would be seen that the total energy at large wavenumbers would be greater than or equal to the kinetic energy. We now examine the energy spectra at each vertical mode number and discuss how decreasing equivalent depths associated with increasing mode numbers affect the mesoscale slopes.\\

\subsection{Modal Analysis}
The energy spectra of the vertical modes exhibit a changing behavior in both the geostrophic and ageostrophic decomposition, as the mode number increases.  We separate discussion into the barotropic and baroclinic modes.\\

\subsubsection{Barotropic Mode}
The barotropic mode spectra has steep slopes in the geostrophic and RKE spectra, at $-3.6$ in the mesoscale for both. The ageostrophic slope is even steeper, at a value of $-4.0$ in the mesoscale. \\

The most interesting feature of the barotropic mode occurs in the DKE spectrum, where the  mesoscale slope is much shallower at $-2.5$ as seen in Figure \ref{fig:barotropicKEPE}. This stark contrast between the ageostrophic and DKE spectra occurs only in the barotropic mode. An interesting feature to note is that the DKE spectrum also contains significantly less energy than the ageostrophic mode, meaning that potential energy is the main contributor to the ageostrophic energy. Indeed, for all but the largest horizontal length scales, the geostrophic and potential energy spectra are in agreement in Figure \ref{fig:barotropicKEPE}, implying that the geostrophic mode has very little APE.

\subsubsection{Baroclinic Modes}
\label{sec:baroclinicanalysis}
We now turn our attention to the remaining vertical modes, the baroclinic modes. Before discussing the mathematical properties of the geostrophic, ageostrophic, and Helmholtz spectra, qualitative properties of the baroclinic modes are examined.\\

We first look at the geostrophic and rotational spectra for the baroclinic modes. For the first several baroclinic modes, the geostrophic mode and RKE spectra align except for the largest horizontal scales. As the vertical mode number increases, the RKE energy at larger horizontal scales begins to separate from the geostrophic energy. By vertical mode 11, the geostrophic spectrum contains more energy compared to RKE spectrum at the larger end of the mesoscale ($6 \leq \tilde{k} < 10$), while they are approximately equal for the smaller end of the mesoscale ($\tilde{k} > 15$). By around vertical mode 25, the geostrophic spectrum contains more energy than the RKE throughout the entire mesoscale, only aligning for horizontal scales corresponding to $<100$ km wavelengths.\\

For the first 10 baroclinic modes, the mesoscale slope of the geostrophic spectrum stays roughly constant, fluctuating around $-3.5$, while the rotational spectrum slope decreases slightly from around $-3.5$ to $-3$ over this same range. From vertical modes 5 through 15, where the mesoscale energy peaks (see Figure \ref{fig:meso_GeoAgeo}), the geostrophic spectrum is steeper than $-3$ throughout, while the RKE mesoscale slope decreases below $-3$ beginning at mode 10. Beyond mode 10, the RKE mesoscale spectrum begins to shallow at a faster rate than the geostrophic spectrum, indicating that potential energy is energizing the mesoscale.\\

The ageostrophic and DKE spectra have slightly different behavior occurring throughout the vertical modes. In the first several vertical modes, the ageostrophic spectrum has more energy than the DKE at all horizontal scales due to the inclusion of potential energy. Vertical mode 2 is especially interesting because of the large increase in DKE mesoscale slope seen in Figure \ref{fig:slopes}.  As the vertical modes increase, the energy in the larger horizontal scales decreases and the mesoscale slopes shallow, similar to the rotational and RKE spectra. However, a difference between the geostrophic/RKE spectra and the ageostrophic/DKE spectra is that the energy in the mesoscale actually increases with increasing vertical mode number for the first 10 vertical modes which contributes to the mesoscale shallowing (see Figure \ref{fig:meso_GeoAgeo}). Beyond mode 10, the energy in the mesoscale of the ageostrophic and DKE spectra starts to decrease again as in the geostrophic and RKE spectra.\\

The ageostrophic and DKE mesoscale spectra both have a sharp spike at vertical mode 2. Excluding the barotropic mode, this also contains the most synoptic scale energy of any vertical mode, as well as relatively small mesoscale energy in comparison. The difference between the very energetic large scale and less energetic mesoscale in vertical mode 2 can help explain the sharp spike at vertical mode 2 for the ageostrophic and DKE in Figure \ref{fig:slopes}, when also taking into account the kinetic and potential energy in the mesoscale shown in Figure \ref{fig:meso_KEPE}. From mode 5 onward, the ageostrophic spectrum continues to be steeper than the DKE spectrum, but it also shallows at a faster rate when compared to the DKE spectrum. Past vertical mode 20, the DKE spectrum becomes steeper than the RKE spectrum.\\

We now provide a mathematical justification for the spectra of the normal mode and Helmholtz decompositions shown in Figures \ref{fig:GeoAgeo_RKEDKE_1-3} - \ref{fig:GeoAgeo_RKEDKE_29}. Equation (\ref{eq:amplitudes}) can be written in terms of the Helmholtz decomposition for a specific $\mathbf{k}$ and $n$ to give

\begin{align}
A^0_{\mathbf{k},n} &= \frac{1}{\sqrt{c^2_n K^2 + f^2}} \left[ c_n (ik\widehat{V} - il\widehat{U}) + f \widehat{\eta} \right],\\
&= \frac{c_n}{\sqrt{c^2_n K^2 + f^2}}\widehat{\zeta} + \frac{f}{\sqrt{c^2_n K^2 + f^2}} \widehat{\eta},\\
&= \frac{\widehat{\zeta}}{\sqrt{K^2 + (f/c_n)^2}} + \frac{(f/c_n) \widehat{\eta}}{\sqrt{K^2 + (f/c_n)^2}}.
\end{align}

The resulting amplitude for the geostrophic mode can then be written 

\begin{align}
|A^0_{\mathbf{k},n}|^2 &= \frac{1}{K^2 + (f/c_n)^2} \left[ \left(\widehat{\zeta}^* + \frac{f}{c_n} \widehat{\eta}^*\right) \left(\widehat{\zeta} + \frac{f}{c_n} \widehat{\eta}\right)\right],\\
&= \frac{1}{K^2 + (f/c_n)^2} \left[ |\widehat{\zeta}|^2 + \left(\frac{f}{c_n}\right)^2 |\widehat{\eta}|^2 + \frac{f}{c_n} \left(\widehat{\eta}^*\widehat{\zeta} + \widehat{\eta}\widehat{\zeta}^*\right)\right]\label{eq:geoBreakdown}.
\end{align}
The ratio $c_n/f$ is essentially the Rossby deformation radius, and so it can be helpful to analyze the geostrophic and ageostrophic spectra at length scales much larger or smaller than the deformation radius. For $ K \gg f/c_n$, as is the case for the first several vertical modes at most values of $K$, then 
\begin{align}
|A^0_{\mathbf{k},n}|^2 \approx \frac{|\widehat{\zeta}|^2}{K^2},
\end{align}
which means that the geostrophic spectrum is approximately equal to the RKE. As the ratio $f/(Kc_n)$ increases, the geostrophic energy has additional contributions from the two right-most quantities in equation (\ref{eq:geoBreakdown}). This is why for the first several vertical modes, the geostrophic and RKE spectra are approximately equal over all horizontal length scales. Indeed, by referring to Figures \ref{fig:Geo_APERKE_1-3} - \ref{fig:Geo_APERKE_29}, the geostrophic mode transitions from being made up of almost entirely RKE in vertical mode 1 (see \ref{fig:Geo_APERKE_1-3}), to being made up of mostly APE for $\tilde{k} < 20$ and mostly RKE for $\tilde{k} > 20$ in vertical mode 11 (see Figure \ref{fig:Geo_APERKE_9-11}). By vertical mode 25 (see Figure \ref{fig:Geo_APERKE_25-27}), the geostrophic mode over the entire mesoscale is mostly from the APE.  As the vertical mode increases, and thus $c_n$ decreases, $K$ must increase for the RKE and geostrophic spectra to align again. This shows that for decreasing equivalent depths, the vortical motion is taking place at smaller horizontal length scales. By vertical mode 29, the geostrophic and RKE spectra do not coincide until $\tilde{k} \approx 100$, meaning much of the vortical motion is taking place at the sub-mesoscale.\\

We can also write the ageostrophic mode in terms of the Helmholtz decomposition, resulting in

\begin{align}
|A^+_{\mathbf{k},n}|^2 & = \frac{1}{2K^2 \alpha^2} \left[\left( -f \widehat{\zeta}^* - i\alpha \widehat{\delta}^* + c_n K^2 \widehat{\eta}^* \right)\left( -f \widehat{\zeta} + i\alpha\widehat{\delta} + c_n K^2 \widehat{\eta}\right)\right]\\
&= \frac{1}{2K^2\alpha^2} \left ( \left[f^2 |\widehat{\zeta}|^2 + \alpha^2 |\widehat{\delta}|^2 + c^2_n K^4 |\widehat{\eta}|^2\right] + if\alpha \widehat{\zeta}^* \widehat{\delta} + i \alpha c_n K^2 \widehat{\delta}^* \widehat{\eta} - f c_n K^2 \widehat{\zeta}^* \widehat{\eta} + \hbox{c.c.} \right),
\end{align}
where $\alpha^2 = c^2_nK^2 + f^2$ is used for notational simplicity and $\hbox{c.c.}$ means complex conjugate. This can be written in terms of the ratio $f/c_n$ 
\begin{align}
\begin{split}
|A^+_{\mathbf{k},n}|^2  &= \frac{1}{2K^2} \frac{ (f/c_n)^2}{K^2 + (f/c_n)^2} |\widehat{\zeta}|^2 + \frac{1}{2K^2} |\widehat{\delta}|^2 + \frac{1}{2\left(1 + \left(\frac{f}{c_nK}\right)^2\right)} |\widehat{\eta}|^2\\
&~+ \frac{i(f/c_n)}{2K^2 \sqrt{K^2  + (f/c_n)^2}} \widehat{\zeta}^* \widehat{\delta} + \frac{i}{2\sqrt{K^2 + (f/c_n)^2}} \widehat{\delta}^* \widehat{\eta} - \frac{f/c_n}{2\left(K^2 + (f/c_n)^2 \right)} \widehat{\zeta}^* \widehat{\eta} + \hbox{c.c.}, \label{eq:ageoPlusBreakdown}
\end{split}
\end{align}
For the other ageostrophic mode, $A^-_{\mathbf{k},n}$, the result is very similar:
\begin{align}
\begin{split}
|A^-_{\mathbf{k},n}|^2  &= \frac{1}{2K^2} \frac{ (f/c_n)^2}{K^2 + (f/c_n)^2} |\widehat{\zeta}|^2 + \frac{1}{2K^2} |\widehat{\delta}|^2 + \frac{1}{2\left(1 + \left(\frac{f}{c_nK}\right)^2\right)} |\widehat{\eta}|^2\\
&~- \frac{i(f/c_n)}{2K^2 \sqrt{K^2  + (f/c_n)^2}} \widehat{\zeta}^* \widehat{\delta} - \frac{i}{2\sqrt{K^2 + (f/c_n)^2}} \widehat{\delta}^* \widehat{\eta} - \frac{f/c_n}{2\left(K^2 + (f/c_n)^2 \right)} \widehat{\zeta}^* \widehat{\eta} + \hbox{c.c.}, \label{eq:ageoMinusBreakdown}
\end{split}
\end{align}
such that 
\begin{align}
|A^0_{\mathbf{k},n}|^2 + |A^-_{\mathbf{k},n}|^2 + |A^-_{\mathbf{k},n}|^2 = |\widehat{U}|^2 + |\widehat{V}|^2 + |\widehat{\eta}|^2.
\end{align}

 For $K \gg f/c_n$,  the ageostrophic spectrum can be approximated by
\begin{align}
|A^+_{\mathbf{k},n}|^2 \approx \frac{1}{2K^2} |\widehat{\delta}|^2 + \frac{1}{2} |\widehat{\eta}|^2 + \frac{i}{2K} \widehat{\delta}^* \widehat{\eta} + \hbox{c.c.}, \label{eq:ageoLargeScales}
\end{align}
 interpreted as being made up from the DKE and the available potential energy, with cross-terms between the two also appearing. As the vertical modes increase and $c_n$ decreases, the contribution to the ageostrophic mode from the APE decreases.\\

 In vertical mode 1 (see Figure \ref{fig:Ageo_APEDKE_1-3}), the ageostrophic spectrum is made up almost entirely of APE at all but the largest horizontal length scales. Equation (\ref{eq:ageoLargeScales}) shows that the DKE contribution is independent of the ratio $f/Kc_n$. Therefore, the full DKE energy is contained in the ageostrophic mode, but the DKE contains about an order of magnitude less than the APE throughout the mesoscale. The RKE contribution for vertical mode 1 is negligible for the mesoscale, since $K \gg f/c_n$. \\
 
In vertical mode 11 (see Figure \ref{fig:Ageo_APEDKE_9-11}), the APE spectrum contains more energy than the ageostrophic spectrum for $\mathbf{\tilde{k}} < 30$. In this regime, the ratio $f/Kc_n$ is no longer negligible, and the APE contribution to the ageostrophic mode decreases, while the RKE contribution to the ageostrophic mode increases. By $\mathbf{\tilde{k}} > 40$, the ageostrophic spectrum crosses the APE spectrum. Here $f/Kc_n$ is becoming smaller and the ageostrophic spectrum is once again well approximated by equation (\ref{eq:ageoLargeScales}).\\
 
 In vertical mode 25, the APE and ageostrophic spectra do not cross until around $\mathbf{\tilde{k}} \approx 60$, meaning that the DKE contribution to the ageostrophic spectrum is important in determining the mesoscale slope, despite that the DKE spectrum contains several orders of magnitude less energy than the APE spectrum.\\
 
Figure \ref{fig:totalGeoAgeo} becomes easier to understand when combining all of the above along with Figures  \ref{fig:modal_energy} and \ref{fig:modal_energy_nobaro}, where it can be seen that a majority of the total energy is contained in the first 6 vertical modes. \\

For the first 5 modes, where the equivalent depth is relatively large, the geostrophic spectrum is composed of mostly RKE in the mesoscale. Beginning at mode 6, the geostrophic spectrum is a blend of the RKE and APE in the large end of the mesoscale ($6 < \tilde{k} < 10$), while the small end of the mesoscale ($10 < \tilde{k} < 60$) remains dominated by RKE. This trend continues for increasing vertical modes, with the APE becoming increasingly important at small wavenumbers, while at the same time the APE influence expands to higher horizontal wavenumbers. By mode 15, the mesoscale is strongly determined by the APE and the RKE only dominates at microscales.\\

The make up of the ageostrophic portion of motion is a more intricate combination of the APE, DKE, and RKE. In the barotropic mode, where $f/Kc_n$ is negligible throughout the mesoscale, the ageostrophic spectrum is dominated by APE since the DKE is orders of magnitude smaller. For the first 5 vertical modes, RKE plays an increasingly important role in the large horizontal scales, while the APE contribution decreases from the increasing $f/Kc_n$ ratio. Modes 4 and 5 have a shallower slope in the mesoscale when compared to the APE, due to the increased importance of the shallower spectra of the DKE and RKE. By mode 11, the APE spectrum contains more energy than the ageostrophic spectrum throughout the entire mesoscale. \\

The APE contribution at each vertical mode for both the geostrophic and ageostrophic spectra explains the mesoscale slopes in Figure \ref{fig:totalGeoAgeo}. The geostrophic spectrum is for the most part similar to the RKE spectrum, but slightly steeper due to the increasing APE contribution to the mesoscale spectrum as the vertical modes increase. The steepness of the ageostrophic mesoscale spectrum when compared to the DKE spectrum can be explained by the importance of the APE in the first few vertical modes. As the DKE and RKE become more significant in the ageostrophic energy around vertical mode 5, the ageostrophic spectrum shallows.
\newpage
\section{Conclusions}
The shallowing of the energy spectrum at the mesoscale has been well-observed in atmospheric turbulence and plays an important role in the energy budget, connecting large scales containing vast amounts of energy to the microscales where dissipation occurs. Understanding the underlying processes of the mesoscale shallowing can help provide insight into the turbulent cascade of energy in the mesoscale.\\

The Helmholtz decomposition is commonly used on the horizontal velocity fields to separate a rotating, stratified fluid into vortical and gravity wave motion. The resulting solenoidal and irrotational fields are then crudely used as a proxy for the vortical motion and gravity waves. The Helmholtz decomposition, while simple, does not include the effects of potential energy in the two dominant modes. Furthermore, the balanced vortical mode has small, but non-zero, divergence, which is not present in the solenoidal component of the Helmholtz decomposition. Similarly, rotational energy in inertia-gravity waves is omitted by the irrotational component, as the name suggests. \\

In our work, we presented an improved decomposition over the Helmholtz decomposition. By first decomposing the vertical structure of the baroclinic jet into into normal modes, we obtain geostrophic and ageostrophic modes that offer a more physically realistic decomposition of the flow compared to the Helmholtz decomposition. However, our normal mode decomposition is still an approximation to the true dynamics taking place in the atmosphere. For example, the set of equations used in the derivation of the vertical structure make use of the hydrostatic approximation, which is not valid across all scales. A normal mode decomposition that considers a non-hydrostatic set of equations would of course be more accurate, but the hydrostatic approximation greatly simplifies the underlying mathematics and the simulation studied in this thesis remains fairly hydrostatic. Though the geostrophic mode in our approach is still divergence-free like the Helmholtz decomposition, the ageostrophic mode contains rotational kinetic energy as well as potential energy which affects the shape of the energy spectrum.\\

Furthermore, the normal mode decomposition allows an extension of the discussion to explicitly include not only horizontal length scales, like the Helmholtz decomposition, but also vertical scales as a result of the eigenvalues in the vertical structure equation. These eigenvalues turn out to be important in determining the overall shape of the geostrophic and ageostrophic spectra.\\

In terms of the energy contained in the vertical modes, the most energy was contained in the first 6 vertical modes. However, most of this energy is in the synoptic scale. If we restrict our attention to the mesoscale, the most energetic modes actually range from around vertical mode 5 to vertical mode 15. For these modes, the equivalent depths from the eigenvalue problem are small enough that the ratio $f/Kc_n = \mathcal{O}(1)$ at the mesoscale. In this regime, complicated contributions from the APE,  DKE, and RKE make up the geostrophic and ageostrophic modes.\\
 
By summing over all of the vertical normal modes, the geostrophic and ageostrophic spectra can be compared to the Helmholtz decomposition. In general, we find the mesoscale slope of the geostrophic spectrum to be steeper than the RKE spectrum, with values of $-3.1$ and $-2.7$, respectively. In the vertical mode numbers where the mesoscale energy is the highest, the increased importance of the APE on the geostrophic mode causes the spectrum to steepen, shifting it away from the RKE spectrum.\\

The difference between the normal mode decomposition and the Helmholtz decomposition is most visible when examining the ageostrophic and DKE spectra. While the RKE and geostrophic mode contain similar amounts of energy over the entire spectrum, the ageostrophic mode contains almost an order of magnitude more energy than the DKE at every wavenumber up until the sub-mesoscale. The mesoscale slopes of the ageostrophic and DKE spectra also have a much starker contrast compared to the geostrophic and RKE spectra, with mesoscale slopes of $-2.7$ for the ageostrophic mode and $-1.9$ for the DKE. The inclusion of APE and, to a lesser extent, RKE is the main source of the difference between the DKE and ageostrophic mesoscale slopes. The ageostrophic mode is comprised of mainly DKE and APE when the ratio $f/Kc_n$ is small, with the APE contribution becoming less important and RKE becoming more important as $f/Kc_n$ increases. In the vertical modes where mesoscale energy is the highest, the transition of the ageostrophic spectrum from being mostly influenced by APE to being mostly influenced by DKE and RKE takes place around the middle of the mesoscale. This transition leads to the mesoscale shallowing seen in the ageostrophic mode, but the APE prevents it from completely agreeing with the DKE.\\

