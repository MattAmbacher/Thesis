% T I T L E   P A G E
% -------------------
% Last updated May 24, 2011, by Stephen Carr, IST-Client Services
% The title page is counted as page `i' but we need to suppress the
% page number.  We also don't want any headers or footers.
\pagestyle{empty}
\pagenumbering{roman}

% The contents of the title page are specified in the "titlepage"
% environment.
\begin{titlepage}
        \begin{center}
        \vspace*{1.0cm}

        \Huge
        {\bf Normal Mode Wave-Vortex Decompositions of Mesoscale Simulations}

        \vspace*{1.0cm}

        \normalsize
        by \\

        \vspace*{1.0cm}

        \Large
        Matthew R. Ambacher \\

        \vspace*{3.0cm}

        \normalsize
        A thesis \\
        presented to the University of Waterloo \\ 
        in fulfillment of the \\
        thesis requirement for the degree of \\
        Master of Mathematics \\
        in \\
        Applied Mathematics \\

        \vspace*{2.0cm}

        Waterloo, Ontario, Canada, 2017 \\

        \vspace*{1.0cm}

        \copyright\ Matthew R. Ambacher 2017 \\
        \end{center}
\end{titlepage}

% The rest of the front pages should contain no headers and be numbered using Roman numerals starting with `ii'
\pagestyle{plain}
\setcounter{page}{2}

\cleardoublepage % Ends the current page and causes all figures and tables that have so far appeared in the input to be printed.
% In a two-sided printing style, it also makes the next page a right-hand (odd-numbered) page, producing a blank page if necessary.
 


% D E C L A R A T I O N   P A G E
% -------------------------------
  % The following is the sample Delaration Page as provided by the GSO
  % December 13th, 2006.  It is designed for an electronic thesis.
  \noindent
I hereby declare that I am the sole author of this thesis. This is a true copy of the thesis, including any required final revisions, as accepted by my examiners.

  \bigskip
  
  \noindent
I understand that my thesis may be made electronically available to the public.

\cleardoublepage
%\newpage

% A B S T R A C T
% ---------------

\begin{center}\textbf{Abstract}\end{center}

The kinetic energy spectrum of the atmosphere has been well observed to exhibit a $k^{-3}$ power law at synoptic scales with a transition to a shallower $k^{-5/3}$ power law at the mesoscale.  To better understand the mesoscale kinetic energy spectrum, the spectrum can be decomposed into the two dominant modes at this scale: quasi-horizontal vortex motion and inertia-gravity wave motion.  A commonly used technique for this is a Helmholtz decomposition of the horizontal velocity into rotational and divergent components, representing the geostrophically balanced and inertia-gravity wave modes, respectively.  This decomposition is a crude approximation, since geostrophically balanced flows have small but non-zero divergence and inertia-gravity waves can have non-zero rotational energy.\\

We investigate the mesoscale spectrum generated in a moderate-resolution, doubly-periodic, non-hydrostatic simulation of a baroclinically unstable jet.  A three-dimensional normal mode decomposition is used to decompose the total energy into geostrophic and ageostrophic components. We compare the results with the Helmholtz decomposition and find they are qualitatively similar, but ageostrophic modes increasingly dominate with increasing vertical wave number. Specifically, we find the geostrophic mode in the mesoscale to have a steeper spectral slope than the spectral slope of the rotational component of the Helmholtz decomposition at $-3.1$ and $-2.7$, respectively. The difference between the spectral slopes of the ageostrophic and divergent modes are much greater, with mesoscale slope values of $-2.7$ and $-1.9$, respectively. We find that the reason for these differences can be attributed to both the inclusion of the available potential energy in the normal mode decomposition, and the inclusion of rotational energy in the ageostrophic mode of the normal mode decomposition.

\cleardoublepage
%\newpage

% A C K N O W L E D G E M E N T S
% -------------------------------

\begin{center}\textbf{Acknowledgements}\end{center}

First and foremost, I want to thank my supervisor, Mike Waite. Without his help and guidance, this thesis would have never seen the light of day. His availability to meet with me, give feedback, and provide thoughtful discussion (many times of which were on short notice) was integral in finishing this thesis on time. Mike has truly made my two years here an enjoyable time.\\

I would also like to acknowledge my committee members, Francis Poulin and Marek Stastna, for their help over the last two years. Francis' help to better understand the normal mode theory was crucial to setting up simulations and piecing together the results. I want to thank Marek not only for his help and support during my Master's degree, but also going back to my time as an undergrad, where he got me interested in fluid mechanics. \\

Lastly, I want to thank all of the guys in my office: Shawn Corvec, Jonathan Drake, Anthony Caterini, and Tony-Pierre Kim. Thanks for keeping me sane over the past two years.  
\cleardoublepage
%\newpage

% D E D I C A T I O N
% -------------------

\begin{center}\textbf{Dedication}\end{center}

\noindent To my Aunt Maureen, who always encouraged me to pursue the various interests in my life. I wish you were here to read this thesis.

\cleardoublepage
%\newpage

% T A B L E   O F   C O N T E N T S
% ---------------------------------
\renewcommand\contentsname{Table of Contents}
\tableofcontents
\cleardoublepage
\phantomsection
%\newpage

% L I S T   O F   T A B L E S
% ---------------------------
\addcontentsline{toc}{chapter}{List of Tables}
\listoftables
\cleardoublepage
\phantomsection		% allows hyperref to link to the correct page
%\newpage

% L I S T   O F   F I G U R E S
% -----------------------------
\addcontentsline{toc}{chapter}{List of Figures}
\listoffigures
\cleardoublepage
\phantomsection		% allows hyperref to link to the correct page
%\newpage

% L I S T   O F   S Y M B O L S
% -----------------------------
% To include a Nomenclature section
% \addcontentsline{toc}{chapter}{\textbf{Nomenclature}}
% \renewcommand{\nomname}{Nomenclature}
% \printglossary
% \cleardoublepage
% \phantomsection % allows hyperref to link to the correct page
% \newpage

% Change page numbering back to Arabic numerals
\pagenumbering{arabic}
\newpage
